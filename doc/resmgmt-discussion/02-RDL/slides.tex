
\documentclass{beamer}
\usepackage{graphicx}
\usepackage{verbatim}
\usepackage{listings-lua}
\usepackage{tikz}
\usepackage{color}
\usepackage{amsmath}

\lstset{basicstyle=\footnotesize\ttfamily}

\mode<presentation> {
\usetheme{Madrid}
%\usecolortheme{default}
%\setbeamertemplate{navigation symbols}{} 
}

% TikZ setup:
\usetikzlibrary{arrows,shapes,positioning,shadows,trees}
\begin{document}

\title[GRR, ARGH. Session 2]{Generic Resource Representation,\\
                  Abstract Resource Grouping Hierarchy \\
                  (GRR, ARGH) \\ 
                  Session 2}
\subtitle{Concepts for Flux Resource Description Language}
\author[Grondona]{Mark A. Grondona}
\institute{LLNL}
\date{\today}


% Title slide:
\maketitle

\begin{frame}{Resource Selection}{Brief Survey Report}
\begin{itemize}
 \item Matchmaking
 \begin{itemize}
  \item {Classified Advertisements (ClassAd)}
  \item {Constraint Language Approach: RedLine}
  \item {Virtual Grid Description Language: vgDL}
 \end{itemize}
 \item Other systems
 \begin{itemize}
  \item {SLURM}
  \item {OAR}
  \item {VXDL}
 \end{itemize}
 \item High-level concepts for Flux Request Specification Language
\end{itemize}
\end{frame}

%
\begin{frame}{Matchmaking}{Classified Advertisements (classAd)}
\begin{itemize}
  \item<+-> Symmetric - resource {\em advertisers} and {\em requesters} use
          the same semi-structured language.
  \item<+-> Describes a mapping from names to expressions.
  \item<+-> Simple matchmaking compares two ClassAds and all attribute
            expressions must evaluate to true.
\end{itemize}
\end{frame}


\begin{frame}[fragile]{classAd Example}{Host classAd}
\begin{tikzpicture}[remember picture, overlay]
\node [fill=purple!32!blue!12,
       rectangle,
       rounded corners=5pt,
       draw, thick, drop shadow, opacity=1.0] (code) at (current page.center) {%
\begin{minipage}[b][15em][t]{.95\textwidth}
\begin{lstlisting}
  Type = "machine";
  Memory = 64;
  OpSys = "RHEL6.5";
  Name =  "node100";
  Untrusted = { bob, alice, stuart };
  Constraint = 
    !member(other.Owner, Untrusted)
    : DayTime < 8*60*60 || DayTime > 18*60*60;
\end{lstlisting}
\end{minipage}
};
\node [fill=blue!28, rectangle, rounded corners, draw, thick] at (code.north) {%
host classAd};
\end{tikzpicture}
\end{frame}

\begin{frame}[fragile]{classAd Example}{request classAd}
\begin{tikzpicture}[remember picture, overlay]
\node [fill=purple!32!blue!12,
       rectangle,
       rounded corners=5pt,
       draw, thick, drop shadow,
       opacity=1.0] (code2) at (current page.center) {%
\begin{minipage}[b][15em][c]{.95\textwidth}
\begin{lstlisting}
  Type = "job";
  Owner = "Sal";
  Rank = other.memory/32
  Constraint = 
    other.Type == "Machine"
    && Arch == "x86"
    && OpSys == "RHEL6.5"
    && other.Memory >= self.Memory;
\end{lstlisting}
\end{minipage}
};
\node [fill=blue!28, rectangle, rounded corners, draw, thick] at (code2.north) {%
request classAd};
\end{tikzpicture}
\end{frame}

\begin{frame}{Matchmaking}{RedLine}
\begin{itemize}
 \item<+-> New approach to {\em symmetric matching}
 \item<+-> Reinterpret matching as a constraint satisfaction problem
 \item<+-> Describe and match resources whose properties are sets or ranges
 \item<+-> Policy {\em and} priority matching
 \item<+-> Support for resource sets
\end{itemize}
\end{frame}

\begin{frame}{Matchmaking}{RedLine Language Features}
\begin{itemize}
 \item<+-> Requests and resources are expressed by {\em descriptions}
 \item<+-> Requestors and resource providers associate common meaning to
           variables ({\em attribute variables})
 \item<+-> Requests for a single resources expresses a set of constraints
 \item<+-> Multiple resources are requested by associating variables with
           description via operators to create {\em resource variables}:
 \begin{itemize}
   \item<+-> `$\langle\text{\em variable}\rangle\ \text{\tt ISA}\ \langle\text{\em description}\rangle$'  or
   \item<+-> `$\langle\text{\em variable}\rangle\ \text{\tt ISA SET}\ \langle\text{\em description}\rangle$'
 \end{itemize}
 \item<+-> Where {\tt ISA} treats resource matches as value domain for
           {\em variable}, and {\tt ISA SET} uses {\em all} matching
           resources as {\em candidate sets} for {\em variable}.
\end{itemize}
\end{frame}

\begin{frame}{Matchmaking}{RedLine Language Features}
\begin{itemize}
 \item<+-> Logic expressions
 \item<+-> {\em Forall x in }$\langle\text{\em set}\rangle$
 \item<+-> {\em Forany x in }$\langle\text{\em set}\rangle$
 \item<+-> {\em Required}$\langle\text{\em set of attributes}\rangle$
 \item<+-> {\em Maximize, Minimize, Search\#, Distribute, SetConstruct}
 \item<+-> Set Functions:
 \begin{itemize}
  \item<+-> Cardinality({\em set})
  \item<+-> Max({\em set}) and Min({\em set})
  \item<+-> Sum({\em set})
  \item<+-> InSet({\em set, value})
  \item<+-> SetIntersection({\em set}), SetUnion({\em set}), SetDifference({\em set})
 \end{itemize}
\end{itemize}
\end{frame}

\begin{frame}[fragile]{RedLine Example}{Resource {\em Description}}
\begin{tikzpicture}[remember picture, overlay]
\node [fill=purple!32!blue!12,
       rectangle,
       rounded corners=5pt,
       draw, thick, drop shadow,
       opacity=0.8] (code1) at (current page.center) {%
\begin{minipage}[t][16em][c]{.95\textwidth}
\begin{lstlisting}
R1=[type="computation"; hn="hype1.llnl.gov";
    cpuspeed=2800;
    bandwidth=DICTIONARY[ {"hype2.llnl.gov", 20}, ...,
                          {"hype100.llnl.gov",10} ];
    accesstime > 17; ] 
R2=[type="computation"; hn="hype2.llnl.gov";
    cpuspeed=2800;
    bandwidth=DICTIONARY[ {"hype1.llnl.gov", 20}, ...,
                          {"hype100.llnl.gov",10} ];
    accesstime > 17; ] 
R2=[type="computation"; hn="hype2.llnl.gov";
R3=[type="storage"; hn="e1.llnl.gov"; space=100]
R4=[type="storage"; hn="e2.llnl.gov"; space=200]
\end{lstlisting}
\end{minipage}
};
\node [fill=blue!28, rectangle, rounded corners, draw, thick] at (code2.north) {%
RedLine Resource Description};
\end{tikzpicture}
\end{frame}

%
%
%
\begin{frame}[fragile]{RedLine Example}{Request {\em Description}}
\begin{tikzpicture}[remember picture, overlay]
\node [fill=purple!32!blue!12,
       rectangle,
       rounded corners=5pt,
       draw, thick, drop shadow,
       opacity=0.8] (code1) at (current page.center) {%
\begin{minipage}[t][16em][c]{.95\textwidth}
\begin{lstlisting}
[
  user="name",
  Group="groupname",
  computation ISA SET[
    type="computation",
    cpuspeed > 150; accesstime > 18;
    SetConstruct Greedy memory];
  storage ISA [
    type="storage";
    space > 100; accesstime > 18];
  Forall x in computation;
    x.bandwidth[storage.hn] > 30;
  Sum(computation.memory) > 300;
  Search #1;
  Distribute FF
]
\end{lstlisting}
\end{minipage}
};
\node [fill=blue!28, rectangle, rounded corners, draw, thick] at (code2.north) {%
RedLine Request Description};
\end{tikzpicture}
\end{frame}

% "Efficient Resource Description and High Quality Selection for Virtual Grids"
\begin{frame}{Matchmaking}{Virtual Grid Description Language (vgDL)}
\begin{itemize}
 \item {\em Application} oriented resource description language
 \item Features:
 \begin{itemize}
  \item Resource aggregates
  \item Network connectivity
  \item Composition of aggregates
  \item Ranking (preference programming)
 \end{itemize}
 \item High-level, qualitative resource {\em Aggregation} operators:
 \begin{description}
  \item[LooseBag] Loosely connected collection of heterogeneous resources.
  \item[TightBag] Tightly connected collection of heterogeneous resources.
  \item[Cluster] Tightly connected collection of homogeneous resources.
 \end{description}
 \item Coarse, qualitative network connectivity operators
 \begin{description}
  \item[close/far] Network proximity latency
  \item[highBW/lowBW] Network bandwidth
 \end{description}
\end{itemize}
\end{frame}

\begin{frame}{vgDL, Cont\ldots}
\begin{itemize}
 \item Allows arbitrary {\em ranking function}
 \item Properties and constraints use {\em RedLine} syntax
 \item Focus on operations at a high, qualitative level
 \begin{itemize}
  \item terms such as {\em close} and {\em far} are intentionally vague
 \end{itemize}
 \item Mainly a research project
 \item Unclear if anything done recently with vgDL
\end{itemize}
\end{frame}

\begin{frame}[fragile]{vgDL Examples}
\begin{itemize}
 \item<+-> Simple request for a {\em Cluster} of nodes: \\
 \vskip .5em
 \colorbox{gray!20}{%
 \lstinline{X = ClusterOf<Node1>[8:32]; Node1 = {C1, C2, ...} at <Date>}} \\
 \vskip .5em
 ({\em Cluster} of nodes of type {\tt Node1} with constraints $C_1,\ldots,C_n$)
 \vskip .5em
 \item<+-> More complex parallel workers example
 \vskip .5em
\begin{lstlisting}[%
  backgroundcolor=\color{gray!20},
  basicstyle=\footnotesize\ttfamily,
]
X = LooseBagOf<LumpedNode>[1:32];
LumpedNode = (SNode = {has database} far LooseBagOf<CNode>[8:128]
              CNode = {memory >= 1GB})

Rank(CNode) = cpu;
Rank(LumpedNode) = count(LumpedNode[]);
\end{lstlisting}
\end{itemize}
\end{frame}

% http://www.sc-camp.org/2010/_pdf/RJMS_OAR_sccamp2010.pdf
% http://oar.imag.fr/sources/2.5/docs/documentation/OAR-DOCUMENTATION-USER.pdf
\begin{frame}{Additional Examples}{OAR}
\begin{itemize}
 \item Hiearchical resources, hieararchical {\em expressions}
 \item Generic resources (e.g. licenses, bandwidth/storage capacity)
 \item Customizable {\em task types} (besteffort, timesharing, idempotent, \ldots)
 \item {\em Container} jobs
\end{itemize}
\end{frame}


\begin{frame}{Additional Examples}{OAR Hiearchical Expressions}
\begin{itemize}
 \item<+-> Standard hiearchical request: \\
 \vskip .5em
 \colorbox{gray!20}{\lstinline$oarsub -l switch=1/nodes=2/cpu=2/core=2 ./app$} \\
 \vskip .5em
  ($1 \times 2 \times 2 \times 2 = 8$ cores) 
 \vskip .5em

 \item<+-> Request 2 nodes entirely \\
 \vskip .5em
 \colorbox{gray!20}{\lstinline{oarsub -l /nodes=2}}
 \vskip .5em

 \item<+-> {\em Moldable} job (multiple {\tt -l} treated as alternatives) \\
 \vskip .5em
 \colorbox{gray!20}{%
  \lstinline{oarsub -l switch=1/nodes=2/cpu=2/core=2}} \\
  \colorbox{gray!20}{%
  \hphantom{oarsub }\lstinline{-l switch=2/nodes=4/cpu=1/core=2}} \\
 \vskip .5em
  OAR runs the alternative that is projected to complete first.
\end{itemize}
\end{frame}

\begin{frame}{Additional Examples}{OAR Continued\ldots}
\begin{itemize}
 \item<+-> Non-uniform requests ($+$ as a {\em sub-reservation} AND) \\
 \vskip .5em
 \colorbox{gray!20}{%
 \lstinline{oarsub -l /cluster=1/nodes=2/core=1+/switch=1/nodes=2/cpu=1}} \\
 \vskip .5em
  Reserve 1 core on 2 nodes on the same cluster AND
  1 cpu on 2 nodes on the same switch.
 \item<+-> OAR request using properties:
 \vskip .5em
 \colorbox{gray!20}{%
 \lstinline$oarsub -l \{memnode=4096 and ib10g=1\}/cluster=1/nodes=2/core=1$
 }
 \vskip .5em
 1 core on 2 nodes on the same cluster with infiniband 10g and 4096 MB memory
\end{itemize}
\end{frame}

\begin{frame}{Additional Examples}{Virtual Resources and Interconnections Description Language (VXDL)}
\begin{itemize}
 \item Specifies and interconnection of virtual resources
       into a virtual infrastructure
 \item Unifies resource description with network topology
 \item Infrastructure forms a graph of interconnected resources
 \item Includes concept of a {\em timeline}
\end{itemize}
\end{frame}

\begin{frame}[fragile]{VXDL Example}
\begin{lstlisting}[%
  backgroundcolor=\color{gray!20},
  basicstyle=\footnotesize\ttfamily,
]
virtual grid Query_With_Latency_Specification {
  nodes (Cluster_NAS) {
    function computing
    size (16, 20)
    node (Nodes_Cluster_NAS) {
       parameters memory_ram (min 256MB),
       cpu_frequency(min 1GHz)
    }
  }
}

virtual topology Query_With_Latency_Specification {
  link (Intra_Cluster) {
    latency (max 10ms),
    between [(Nodes_Cluster_NAS,
              Nodes_Cluster_NAS)]
  }
}
\end{lstlisting}
\end{frame}

\begin{frame}{Flux Resource Specification (or RDL)}{Goals}
\begin{itemize}
 \item Borrow good ideas from RedLine, vgDL, and OAR
 \item However, Flux is not a research project so we need something
        that is practical
 \item Balance power, expressiveness and flexibility with simplicity and
        usability.
 \item Shouldn't be difficult to surpass the status quo for ``traditional''
        resource manager resource request syntax
\end{itemize}
\end{frame}

\begin{frame}{Flux Resource Specification}{High-level proposal}
\begin{itemize}
 \item<+-> Composite Resource Request (symmetric to Composite Resource Description)
 \item<+-> Resource requests are built using same language as resources configuration
 \item<+-> Requests shall also support timespecs, arbitrary constraints, and
       even preferences
 \item<+-> Support registration of ranking function to choose between resources in a set of matches
\end{itemize}
\end{frame}


\begin{frame}[fragile]{Flux Resource Specification}{Examples}
\begin{itemize}
 \item<+-> OAR-style simple request for whole nodes
\vskip .25em
\begin{lstlisting}[%
  backgroundcolor=\color{gray!20},
  basicstyle=\footnotesize\ttfamily,
]
{nodes = 4}
\end{lstlisting}
\vskip .25em
Implies a list of 4 nodes, {\em conjunctive} allocation, e.g.
\begin{lstlisting}[%
  backgroundcolor=\color{gray!20},
  basicstyle=\footnotesize\ttfamily,
]
 {node=1, { all }}, {node=1, { all }, ...
\end{lstlisting}

 \item<+-> Again, OAR-inspired hiearchical request for 2 cores, 4 nodes, 1 cluster
\vskip .25em
\begin{lstlisting}[%
  backgroundcolor=\color{gray!20},
  basicstyle=\footnotesize\ttfamily,
]
{ cluster = 1, { nodes = 4, { cores = 2 }}}
\end{lstlisting}
\vskip .25em
Similar to OAR, this reads {\em foreach of one cluster find four nodes, for each4 nodes find 2 cores\ldots} \\
\end{itemize} 
\end{frame}

\begin{frame}[fragile]{Flux Resource Descriptions}{Examples}
\begin{itemize}
 \item<+-> Disjoint requests at any level in the hierarchical request
\vskip .25em
\begin{lstlisting}[%
  backgroundcolor=\color{gray!20},
  basicstyle=\footnotesize\ttfamily,
]
 {cluster = 1, {nodes = 2, { {cores = 1}, {cores = 2} } } }
\end{lstlisting}
\vskip .25em
1 core on 1 node, 2 cores on another node. (unordered sets)

\item<+-> Property constraints may be specified at any level in hiearchical
          request:
\vskip .25em
\begin{lstlisting}[%
  backgroundcolor=\color{gray!20},
  basicstyle=\footnotesize\ttfamily,
]
 {cluster = 1, name = "hype", ... }
 {nodes = 2, tags = { "hyperthread" }, ... }
\end{lstlisting}
\item<+-> Topology constraints might be specified by organizing resources
          with higher level language tools (TBD)
\end{itemize}
\end{frame}

\begin{frame}[fragile]{Flux Resource Descriptions}{Timespec}
\begin{itemize}
 \item<+-> A resource spec implemented in Lua could avail itself of a library
           of methods useful for building and modifying requests
 \item<+-> For example, while requests for a specific number of a resource
           type must be satisfied at a minimum, maximums could be applied
           with a higher level function
\vskip .25em
\begin{lstlisting}[%
  backgroundcolor=\color{gray!20},
  basicstyle=\footnotesize\ttfamily,
]
R = Request{nodes = 1}
R.set_maximum(3)
\end{lstlisting}
\vskip .25em
Set maximum and other similar methods might be implemented by
appending to a set of alternative resource specifications in the request.
(Lists of requests can be either AND or ORed together)
\end{itemize}
\end{frame}


\begin{frame}[fragile]{Flux Resource Descriptions}{Timespec}
\begin{itemize}
 \item<+-> At the least, {\em start, end}, and {\em duration} can be
           assigned to each composite resource request
 \item<+-> Use of relative time keywords can indicate when resources are
           needed relative to the actual start of a job
\vskip .25em
\begin{lstlisting}[%
  backgroundcolor=\color{gray!20},
  basicstyle=\footnotesize\ttfamily,
]
Req = Request{nodes = 1}
Req.duration = 1h
Req2 = Req.append{nodes = 8}
Req2.start = Req.t0 + 10m
\end{lstlisting}
\vskip .25em
Expandable to allow Dong's idea of {\em min/max} duration, start time, etc.
Satisfies Don's request for a ``delay''

\item<+-> Explicit start time implies reservation, \lstinline{start = "now"}
          implies ``immediate'' job, etc.
\item<+-> Most jobs would set a duration or {\em relative} end time.
\end{itemize}
\end{frame}

%\begin{frame}{Plan}
%\visible<+->
%\centering Three target areas of research:
%\begin{enumerate}
% \item<+-> Backend storage of resource configuration (Resource Inventory)
% \begin{itemize}
%   \item<+-> Use CMB KVS for first version?
%   \item<+-> Investigate available databases like MongoDB and Postgres
%   \item<+-> Begin collecting ideas for RI API
% \end{itemize}
% \item<+-> Configuration syntax/language
% \begin{itemize}
%   \item <+-> Experiemnt defining basic resources with Lua or YAML
% \end{itemize}
% \item<+-> Request syntax/language
% \begin{itemize}
%   \item<+-> A discussion topic for the next meeting\ldots
% \end{itemize}
%\end{enumerate}
%\end{frame}

\end{document}
