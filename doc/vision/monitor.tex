\section{Monitoring/Provenance}

FIXME: pull from 2 presentations

\newpage
\subsection{Monitoring/Provenance WBS}

\begin{longtable}{|p{1cm}|p{10.2cm}|p{1cm}|p{1cm}|p{1.8cm}|}\hline
  \textbf{Item} & \textbf{Description}
                & \textbf{Deliv}\footnote{SD = software drop,
                        DR = design review, V = viewgraphs, D = document}
                & \textbf{Weeks} & \textbf{Dep} \\
  \hline
  \hline
  \multicolumn{5}{|l|}{3.1. \textbf{Root File System via Network Block Device}} \\
  \hline
  3.1.1.& Complete 9nbd network block device development, including
          backport upstream 9P transport fixes, refine/test 9nbd resiliency
          code, submit for LKML integration.
        & SD
        & 
        & \\
  \hline
  3.1.2.& Add MUNGE key bootstrap and 9nbd root support to nfsroot
	  dracut module.
        & SD
        & 
        & \\
  \hline
  3.1.3.& Backport zram TRIM support so space can be reclaimed in
          zram-based root file systems.
        & SD
        & 
        & \\
  \hline
  3.1.4.& Investigate container formats, e.g. local file system type/options,
          and methods of attaching metadata to containers, e.g. extended
          attributes.
        & D
        & 
        & \\
  \hline
  3.1.5.& Root file system provisioning support via diod, 9nbd for TOSS 2.1
          and LC production.
        & SD
        & 
        & 3.1.1, 3.1.2, 3.1.3, 3.1.4\\
  \hline
  3.1.6.& Report result of production deployment.
	  Study scalability impact including reboot times, /opt based pynamic.
	  To what extent does using NFS at the back end inhibit scaling?
	  What other issues have come up with this method?
        & V, D?
        & 
        & 3.1.5\\
  \hline
  \multicolumn{5}{|l|}{3.2. \textbf{Decoupled Application Root/Read-only Datasets}} \\
  \hline
  3.2.1.& Develop prototype SPANK plugin that allows users
          to select among available root images.  Manage private file system
          namespace for user, perfrom post-mount configuration, perform uid
          mapping.  Develop strategy for handling mounts such as home
          directories that would be overmounted by a new root.
        & DR
        & 
        & 3.1.5 \\
  \hline
  3.2.2.& Design/prototype tools to enable code teams and/or development
          environment group to create application-specific roots.
          Lorenz interface?  Streamline upgrading app root when kernel/driver
          ABI changes.  Portability versus incorporated config files.
        & DR
        & 
        & \\
  \hline
  3.2.3.& Design/prototype data store for application-specific roots.
          What is the data model?  Need content addressability so the RM
          can hold a provenance ref.  How to manage dataset pile up, archive
          relationship, purge policy, thin copies, dedup?
          Support tool workflow.  Lorenz interface?   Performance?
        & DR
        & 
        & \\
  \hline
  3.2.4.& Implement tools for application-specific root creation.
        & SD
        & 
        & 3.2.2 \\
  \hline
  3.2.5.& Implement data store for application-specific root creation.
        & SD
        & 
        & 3.2.3 \\
  \hline
  3.2.6.& Design/prototype \ngrm\ integration including RM launch of private
          diod daemons, possible comms framework transport integration,
          RM management of dataset references.  Support snapshotting of
          r/w directories into r/o store at job submission?
        & DR
        & 
        & 3.2.3 \\
  \hline
  3.2.7.& Implement \ngrm\ integration.
        & SD
        & 
        & comms, runtime, 3.2.6 \\
  \hline
  \multicolumn{5}{|l|}{3.3. \textbf{Monitoring Database}} \\
  \hline
  3.3.1.& Study available NoSQL database for 100K node scalability
          and appropriate query interface.
          Use offline log data to investigate system diagnostic capability.
          (Gamblin/Mohror HPC Data Analytics FY12 LDRD)
        & V
        & 
        & LDRD \\
  \hline
  3.3.2.& Implement prototype database tied to live log sources.
          Study scalability and develop queries.
          (Gamblin/Mohror HPC Data Analytics FY12 LDRD)
          (See also: Faaland SPLUNK deployment).
        & V
        & 
        & 3.3.1 \\
  \hline
  3.3.3.& Design/prototype access-role based security.
        & DR
        & 
        & 3.3.2 \\
  \hline
  3.3.4.& Design/prototype schemas and queries for reporting
          RAS metrics of interest to center management.
        & DR
        & 
        & 3.3.2 \\
  \hline
  3.3.5.& Design/prototype procedure for sanitizing and releasing data
	  for research study and citation.
        & DR
        & 
        & 3.3.2 \\
  \hline
  3.3.6.& Design/prototype schema for job logs and queries for
          associating job data, system log data, etc..
        & 
        & 
        & RM, 3.3.2 \\
  \hline
  3.3.7.& Implement production monitoring kit.
          Document procedures and practices for extending and using the system.
          (opportunity for multiple publications here) 
        & SD, D
        & 
        & 3.3.2 3.3.3, 3.3.4, 3.3.5, 3.3.6 \\
  \hline
  \multicolumn{5}{|l|}{3.4. \textbf{Monitoring Framework}} \\
  \hline
  3.4.1.& Design/prototype log aggregation and routing to log database
	  based on \ngrm\ comms.
        &
        & DR
        & comms \\
  \hline
  3.4.2.& Design/prototype event-based monitoring framework based on
	  \ngrm\ comms.  (Coordinate with Meier monitoring improvement project)
        &
        & DR
        & comms \\
  \hline
  3.4.3.& Design/prototype plugin interface that allows
          users to customize event based monitoring within a job.
        &
        & DR
        & 3.4.2 \\
  \hline
  3.4.4.& Design/prototype plugin interface that allows
          users to customize log generation and aggregation within a job.
        &
        & DR
        & 3.4.1 \\
  \hline
  3.4.5.& Design/prototype a plugin interface for instrumenting jobs
          to gather "implicit provenance" such as file accesses.
        & DR
        & 
        & 3.3.2 \\
  \hline
  3.4.6.& Implement monitoring framework.
          Document procedures for writing plugins and extending the system.
        &
        & SD, D
        & 3.4.1, 3.4.2, 3.4.3, 3.4.4, 3.4.5. \\
  \hline
\end{longtable}

