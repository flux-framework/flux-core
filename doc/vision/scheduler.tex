\section{NGRM Job Scheduler}

This section presents the requirements and design for the job
scheduling component of the \ngrm.

The \ngjs\ is responsible for scheduling computing resources to users'
jobs.  Users submit to the scheduler requests for resources to run
their job.  The scheduler implements management's policy to decide
when and where to allocate the resources for each job.

This section presents the requirements for the \ngjs, a rough design
which meets those requirements, and a work breakdown structure for
developing the scheduler component.

\subsection{Motivation}

Scheduling batch jobs across a collection of networked computing
resources started in the 1990's with Livermore Computing's DPCS (later
known as LCRM).  It received users' job requests, selected a cluster
for each job, then dispatched the job to that cluster's resource
manager.  The Moab Workload Manager which replaced LCRM essentially
provided the same functionality.  And while SLURM provides some grid
functionality, it never matured enough to allow it to replace Moab for
production use.

The \ngjs\ will provide new functionality not available in any
commercial or open source project.  The \ngjs\ will schedule jobs
across resources in a computing center without regard to traditional
cluster boundaries.  A job will be able to request resources
containing a common feature (like connectivity to the same high speed
switch) or fitting within a limited power envelope.

In addition, the \ngjs\ will be designed to be recursive.  It will
have the capability to launch another instance of itself within the
context of a job.  The \ngjs\ will support job priority plugin modules
allowing each recursive instance of the scheduler the ability to run a
different priority plugin.  In so doing, the \ngjs's scheduling
capabilities will range from scheduling all resources in the center to
scheduling jobs on dedicated resources (DATs) to scheduling what were
formerly known as job steps.

Most importantly, the traditional boundaries between a job scheduler
and the resource manager will be redefined under \ngrm.  Instead of a
resource manager that manages every resource of a cluster, the
\ngrmfull\ will be instantiated on the fly by the \ngjs\ and manage
only the resources the scheduler allocates to the job (or recursive
job).

In order to continue to meet the needs of LC users, the \ngjs\ must
continue to provide all the services that Moab currently provides,
but...
\begin{itemize}
  \item More quickly and efficiently
  \item More accurately
  \item More reliably
  \item More easily
  \item More intuitively
  \item More flexibly
  \item More securely
  \item More transparently
  \item And require minimal intervention and oversight
\end{itemize}

\subsection{Requirements}

While a more detailed list of requirements is presented here
<ToBeDone>, the following provides an overview of the functionality
that the \ngjs\ will be expected to deliver.

\subsubsection{Fundamental Requirements}

The following is the most definitive list of scheduling requirements.
A scheduler is not a scheduler unless it can do these:

\begin{itemize}
  \item Maintain current resource inventory including state, health
    and availability of all resources to be scheduled
  \item Receive job allocation requests
  \item Prioritize each job
  \item Schedule each job based on multiple resource requirements
  \item Backfill lower priority jobs whenever possible
  \item Allow for dynamic job growth and reduction
  \item Preempt running jobs to free up resources needed by higher priority jobs
  \item Provide status of all jobs with estimates of when each job will run
  \item Allow for recursive scheduling
\end{itemize}

\subsubsection{Further Scheduler Requirements}

As part of its scheduling duties, the scheduler must also accommodate
user requests and provide versatile administration capabilities.

\begin{itemize}
  \item Respond to requests to hold, modify or cancel jobs
  \item Schedule jobs across the center
  \item Support complex job dependencies
  \item Restrict some operations based on roles
  \item Accept job reservations
  \item Originate and manage resource manager instances
  \item Maintain some job statistics (e.g., resource usage per user
    per account per cluster)
  \item Process live configuration updates
  \item Save complete state and recover fully from a restart
\end{itemize}

\subsubsection{Policy Enforcement}

The scheduler implements the center's policies for providing access to
its computing resources.  As part of this duty, the scheduler must

\begin{itemize}
  \item Reject job submissions for jobs which cannot or will never run
  \item Remove jobs that exceed time limits
  \item Honor limits imposed on users, groups, projects (banks), etc
  \item Honor service level agreements and service quality requests
\end{itemize}

\subsubsection{Organization Components}

The \ngjs\ functionality is broken down into the following components.
However it is designed, it must include the following facilities.

\textbf{Job Submission.} This the facility that receives a user's
request for a job allocation and adds a record of this request to
persistent storage.  Each job submitted must be immediately rejected
if it cannot or is not allowed to run given the job specifications,
the resources available, and any restrictions in place.  Job
submission must be extremely fast and return a job ID in well under a
second of time.  The system shall support hundreds of job submissions
per second.

\textbf{Job Prioritization.}  This the facility for prioritizing jobs
based on potentially multiple factors.  The system shall offer a job
priority plugin framework to allow custom algorithms for determining
job priority.  The priority of each queued job must be continually
recalculated as the queue of jobs and constituent factors are
constantly changing.

\textbf{Job Scheduling.} For each job removed from the prioritized
queue, computing resources must be reserved and eventually allocated.
The collection of resources to schedule must be available from the
\ngrm\ system with the state and status of each resource updated in
real-time.  The scheduler must honor multiple resource requests
simultaneously as it seeks to allocate cores, GPUs, nodes, switches,
bandwidth, power, etc.

Here too, the system shall offer a plugin framework to support custom
algorithms for scheduling jobs to compute resources.  An essential
scheduling algorithm which must be included is backfill scheduling
(lower priority jobs are scheduled to run if they do not delay the
start of higher priority jobs).  In addition, qualities of service must
be implemented in the scheduler such that running jobs can be
preempted if needed to free up resources for more important jobs.
This involves not only selecting the best resources for a job, but
also identifying the set of jobs to preempt when such a policy is
enforced.

The output of a the job scheduling process is a schedule of which jobs
are mapped to which resources over a future, rolling period of time.
A by-product of this schedule is a projected start time for every
queued job that is included in the schedule.

\textbf{Job Dispatching.} As time passes, the allocations described in
the schedule must be created.  Running jobs that exceed their wall
clock limit much be terminated and new jobs must be launched.
Provisions must be made to launch multiple jobs simultaneously (or
nearly simultaneously).

\textbf{Job Status Reporting.} This is the facility for showing the
user the status of their jobs and the job queue.  Job info must be
available immediately after job submission, as it is pending, while it
is running, and afterwards for a period to be determined.  The system
should support multiple status requests at a time and reply with a
second or two.  The system is designed to withstand denial of service
attacks - whether deliberate or accidental.

\textbf{Job and System Management.} This is the facility for manual
intervention: boosting job priorities, modifying job characteristics,
cancelling jobs, etc.  Modifying resource states does not have to be
part of this facility.
