\section{Terminology}

\paragraph{comms session}
An established communication association among a set of nodes that
enables secure, scalable, elastic, and fault-resilient communication
services for an \ngrm\ {\bf instance}.
A comms session is identified by its private DNS name, e.g. s1.\ngrm.
or s1.s1.\ngrm.

\paragraph{control node}
A distinguished node within a {\bf comms session} which holds the master
{\em comms state}, is the root of the session's aggregation/reduction tree,
and which performs gateway functions with the parent session.

\paragraph{instance}
An independent set of resource manager services configured to manage
a {\bf resource allocation}.
Instances are created dynamically and recursively.
We refer to the bootstrap {\em root instance} which contains all resources,
and speak of {\em parent}, {\em child}, and {\em sibling} relations between
instances.
An instance is identified by its {\bf comms session} name, e.g. s1.\ngrm.
or s1.s1.\ngrm.

\paragraph{job}
A unit of work submitted to an {\bf instance}, which will obtain a
{\bf resource allocation}, and be launched in a child instance.
The job request defines a set of {\bf job functions} that will be launched.

\paragraph{job function}
A distributed invocation of application or tool components within an
{\bf instance}.  For example, a running MVAPICH program is one job
function, and an instance of TotalView\textsuperscript{\textregistered}
attached to it is another.
Each type of job function is supported by a {\em job function plugin}.

\paragraph{resource allocation}
A set of resources selected by the scheduler of a target {\bf instance}
to fulfill a resource request.
A child instance is created to manage the resource allocation.
