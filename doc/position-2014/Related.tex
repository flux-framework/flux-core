\section{Related Work}

The Flux initiative grows out of a large body of pre-existing work,
both in the academic and commercial domains.  In the areas of
scheduling and resource management, Flux grew out of a growing need
that was previously serviced by LLNL's own SLURM resource
manager~\cite{Jette02slurm} and proprietary LCRM~\cite{LCRM} job
scheduler.

In designing Flux, we looked at the solutions provided by existing
commercial products like IBM's Platform LSF~\cite{LSF}, Adaptive
Computing's Moab~\cite{Moab}, Altair's PBS Professional~\cite{PSBPro},
and Univa's Grid Engine Software~\cite{UnivaGE}.

There are also a number of open source products that provide batch
scheduling and resource management.  In addition to SLURM, the list
includes Globus~\cite{GlobusToolkit},
HTCondor~\cite{Litzkow88,Raman98}, PBS~\cite{PBS},
Maui~\cite{Jackson01corealgorithms}, Mesos~\cite{Mesos} and
Cobalt~\cite{Cobalt}.

There were also historical efforts that offered novel approaches when
they were introduced but are not in widespread use today.  This list
includes NQS~\cite{NQS}, AppLeS~\cite{AppLeS},
Legion~\cite{LegionRM,LegionGrid}, OAR~\cite{Oar} and
OpenCCS~\cite{Keller98ccsresource}.

In all the above, the basic functionality is the same.  Users submit
job allocation requests for computing resources and a scheduler
decides where and when to fulfill the request.  Some schedulers are
optimized for high throughput while others provide high performance.

In various ways, each of the above solutions reached or are reaching
their limits in managing the large and diverse collection of computing
resources being delivered today.  Not only are the size of today's
machines reaching a unprecedented scale, but the collection of
resources which much be orchestrated to service the complex job
dependencies of today is expanding.  In addition, the quantity of jobs
today's computing facilities are being expected to process has grown
considerably.

Flux was initiated to address these needs.  At its core, Flux is
designed to manage all of the computing resources in a center.  Flux's
hierarchical job model is a unique solution that breaks up the
activities of a centralized scheduler and distributes the work to
multiple, smaller schedulers each handling a subset of the workload.
Flux provides a framework that allows for multiple solutions to be
``plugged in''.  Rather than design the best scheduling algorithm in
the world, we recognized the need to allow Flux to offer a range of
solutions, from simple to complex, each suited to their specific
purposes.

This enables Flux to handle a very high job throughput across a large
number of diverse computing clusters and ancillary resources while
providing levels of service that meet the needs of the most demanding
users.
