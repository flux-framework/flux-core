\section{Paradigm Shift in HPC Resource Management}
The vision of FLUX is to create a scalable RM
software system that significantly improves
operational efficiency and user productivity
for the workloads spanning across the entire
computing facility.
With a trend towards ever-growing numbers and
types of compute cores, however, almost all of 
the systems sited at the facility are increasingly
becoming subject to the challenges
that today's capability-class machines face.
\ifcomments
\marginpar{\tiny DA: do we have some stats to back this up?}

The challenges include having to provide extreme
scalability, low noise, fault tolerance, 
and heterogeneity management while under a strict power budget.
Worse, the workloads themselves are also becoming
increasingly diverse, dynamic, and large.
Thus, fully realizing our vision through
these challenges requires a paradigm shift
in how the RM must manage, model, schedule, 
and allocate its resources.

In the new paradigm, the RM must be capable of
imposing highly complex resource bounds
to guarantee the highest operational efficiency
at any level across the computing facility, 
while at the same time enabling most efficient
execution and scheduling of the workloads
within these bounds.
This requires the RM to have a purview over 
the entire computing facility. 
The RM must manage the resources at the center
as one common pool of resources, and the ability
to see a broader spectrum of resources
and their various constraints can then lead to
most efficient scheduling strategies
and execution environments. 

The same ability will ease efforts to diagnose errors
for both end users and support staff by associating jobs
with other facility-wide events. The new paradigm 
also demands that the RM model various types of
resources and their relationships beyond the traditional
resource representation: i.e., a simple collection
of compute nodes. The rich resource model will allow
the RM to allocate computing resources tailored
to the disparate limiting factors
of our applications: e.g., an application
may be compute bound while others are I/O bound 
or power bound.

Under the new paradigm, the resource allocations
must also be elastic. An application may have different
phases with disparate performance-limiting factors;
it must be able to grow and shrink its resource allocation
dynamically.


Finally, the new paradigm must meet the greater difficulties
in code development by facilitating the integration 
of other key relevant software that can ease the difficulties. 
These software components should include system monitoring
and administration, lightweight virtualization,
and scalable tool communication.
The integration will facilitate a higher level of
leverage among these essential computing elements,
and this will lead to significantly higher productivity
for both end users and system administrators.
These capabilities are currently provided
through disjoint and often overlapping software,
the integration will substantially reduce the costs
needed for providing them under the new paradigm. 

In summary, the global resource view, rich resource model,
elasticity, and seamless integration of other software 
represent the fundamental characteristics of the new
 resource management paradigm.  

