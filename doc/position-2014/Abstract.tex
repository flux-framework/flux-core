\begin{abstract}
Resource management (RM) software is crucial to HPC. 
%
However, growing numbers and types of compute resources
and a greater interplay among various resources 
(e.g., between compute clusters and a shared I/O cluster) 
across the entire HPC center makes even 
the best-in-breed RM software increasingly ineffective.
In fact, HPC centers are at a junction where they require
a paradigm shift in managing their resources. 
Our response to this critical need is FLUX, an RM software
framework that can solve the key RM challenges
in a simple, extensible, distributed and autonomous fashion.
It aims at managing the entire computing facility as one common pool
of diverse sets of resources to provide
efficient scheduling decisions and easy accommodation
of emerging site-wide constraints such as increasingly 
stricter power bounds.
Further, FLUX employs a framework approach to facilitate 
the seamless integration of system monitoring and administration,
lightweight virtualization, and parallel programming and tools
run-time systems.
In this paper, we discuss FLUX's vision, design challenges
and concepts, and then report our progress on building
its run-time system.
Our preliminary results show that the run-time  
provides requisite properties such as high scalability
and easy integration and interoperability among essential
run-time elements such as MPI, run-time tools and middleware.
\end{abstract}
